\subsection*{Motivation}
%What problem are you trying to solve? Why is this hard right now?
Deep convolutional neural networks are a leading technique for image classification.
However, \emph{video} classification is less developed.
While it is certainly possible to simply run a neural network on each individual frame of a video, this sacrifices a wealth of temporal attributes such as movement, gestures, gait, etc.
There are indeed methods of preprocessing temporal information into a single 2-dimensional input~\cite{brox}, but a more attractive research goal is to develop a neural network that can discover temporal relationships on its own.
There are multiple recent approaches towards this goal, but as of yet no consensus on which is superior.

The difficulties in training neural networks on video input include the following: memory requirements (particularly if 3D convolutions are used), fewer public data sets, size of the data sets (for instance, the Sports-1M data set is $\bigO(1TB)$ large), and lack of consensus on which approach is most effective. 

Our goal is to develop a deep neural network that classifies videos.
We have settled on two main approaches, discussed below: 3-dimensional convolutional neural networks (3D-CNNs) and recurrent neural networks that incorporate long short-term memory (LSTM).  

\subsection*{Related Work}
%What has been done in similar fields/problems? What are the limitations of current approaches?

%Recurrent long-term convolutional models are most frequently used in speech and language applications, but it is possible to apply them to visual time-series data. 
One popular approach is to apply a 2D-CNN to each frame of the video, followed by an RNN with LSTM~\cite{ltrcn}. 


Large-scale video classification:~\cite{cnnvid}

Learning Spatio-temporal features:~\cite{stf}

Beyond short snippets:~\cite{snip}
\subsection*{Approach and Techniques}
%What is your proposed approach to solving the problem? How does it compare to existing approaches?
%Note: It's OK for these projects to be similar to existing approaches, although if you want to publish the results they will have to have some novelty.

We have decided on combining two possible approaches, which address
the third dimension (time) in different ways: 3D Convolutional Neural
Networks (3D-CNNs) \cite{stf,cnnvid} and Multi-Dimensional Long
Short-Term Memory (LSTM) Recurrent Neural Networks (RNNs) \cite{ltrcn}. 

% 3D-CNNs
Using a 3D-CNNs, it is possible to build a classifier on a moving window
of frames, which extract useful local movement information shared
in a few frame. We would like to use the slow fusion model \cite{cnnvid}
due to the GPU memory constrain. On the other hand, LSTM/ RNNs is
able to build the classifier with long term memory
using much larger number of the frames in the video. 
Combining both 3D-CNNs and LSTM is able to combine the
local movement information with long-term memory,
which can further increase the classification accuracy.

% LSTM
Furthermore, the current LSTM model is built on the fully-connected
layer, which no spatial information is preserved. We would like to
try to plug-in the LSTM at earlier layers to test its performance.
Especially, if applying LSTM directly after the convolution layer,
the input to LSTM model is actually multi-dimentional tensor. To address
this challenge, we would like to apply the multi-dimensional LSTM
\cite{byeon2015scene} to efficiently take advantage of the spatial
information after the convolution layer. 


\subsection*{Data Set}
%What data are you planning to apply your approach to?
\TODO{Describe the video data set...}
\subsection*{Experimental Methodology}
%What specific experiments are you planning on conducting? How are they testing the specific problem you want to solve?
For training we plan to use the Jinx cluster at Georgia Tech. Each node is equipped with 2 nVidia Tesla M2090 ``Fermi'' GPU cards, and CPU nodes with large memory are available.
\begin{itemize}
\item Move 2D-LSTM into CNN
\item Replace 2D-CNN $\to$ 2D-CNN with slow fusion
\item Replace 2D-CNN $\to$ Shallow ResNet
\item Replace 2D-CNN $\to$ Shallow ResNet with slow fusion
\item 2D-CNN incorporated with Optical Flow \emph{and} slow fusion
\end{itemize}
\subsection*{Group Tasking}
%A plan of breaking up the project (who will do what). Note that all team members must make a meaningful/significant contribution (i.e. not just do the reporting/posters).
\begin{itemize}
\item Proposal writing - Steve, Hao, Casey
\item Proposal presentation - Yao
\item Read and understand papers on 2D-LSTM - Everyone
\item Read and understand ResNet papers - Everyone
%\item Read and understand caffe code that implements 2D-CNN and LSTM.
\item Implement LSTM: Yao, Casey
\item Implement CNN: Steve, Hao
\item Preparation of UCF data (incl. optical flow) - Casey
\item Preparation of Sports-101 data - (? \dots tentative)
\end{itemize}
