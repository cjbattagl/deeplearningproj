\subsection*{Motivation}
%What problem are you trying to solve? Why is this hard right now?
Deep convolutional neural networks are currently the leading technique for image classification. However, video classification is a less mature field. Training and testing neural networks on a video source requires greater computational resources and more elaborate neural architectures. Furthermore, there are far fewer publicly available data sets to train on. 

The proposed contribution of our project is to apply a \TODO{3D-CNN that incorporates LSTM \dots}
\subsection*{Related Work}
%What has been done in similar fields/problems? What are the limitations of current approaches?

%Recurrent long-term convolutional models are most frequently used in speech and language applications, but it is possible to apply them to visual time-series data. 
Long-term recurrent convolutional networks have been applied to an even harder problem: that of not only classifying video content, but generating a sentence that describes the scene~\cite{ltrcn}.

Large-scale video classification:~\cite{cnnvid}

Learning Spatio-temporal features:~\cite{stf}

Beyond short snippets:~\cite{snip}
\subsection*{Approach and Techniques}
%What is your proposed approach to solving the problem? How does it compare to existing approaches?
%Note: It's OK for these projects to be similar to existing approaches, although if you want to publish the results they will have to have some novelty.

We have decided on combining two possible approaches, which address
the third dimension (time) in different ways: 3D Convolutional Neural
Networks (3D-CNNs) \cite{stf,cnnvid} and Multi-Dimensional Long
Short-Term Memory (LSTM) Recurrent Neural Networks (RNNs) \cite{ltrcn}. 

% 3D-CNNs
Using a 3D-CNNs, it is possible to build a classifier on a moving window
of frames, which extract useful local movement information shared
in a few frame. We would like to use the slow fusion model \cite{cnnvid}
due to the GPU memory constrain. On the other hand, LSTM/ RNNs is
able to build the classifier with long term memory
using much larger number of the frames in the video. 
Combining both 3D-CNNs and LSTM is able to combine the
local movement information with long-term memory,
which can further increase the classification accuracy.

% LSTM
Furthermore, the current LSTM model is built on the fully-connected
layer, which no spatial information is preserved. We would like to
try to plug-in the LSTM at earlier layers to test its performance.
Especially, if applying LSTM directly after the convolution layer,
the input to LSTM model is actually multi-dimentional tensor. To address
this challenge, we would like to apply the multi-dimensional LSTM
\cite{byeon2015scene} to efficiently take advantage of the spatial
information after the convolution layer. 


\subsection*{Data Set}
%What data are you planning to apply your approach to?
\TODO{Describe the video data set...}
\subsection*{Experimental Methodology}
%What specific experiments are you planning on conducting? How are they testing the specific problem you want to solve?
\TODO{\dots}
For training we plan to use the Jinx cluster at Georgia Tech. Each node is equipped with 2 nVidia Tesla M2090 ``Fermi'' GPU cards.
\subsection*{Group Tasking}
\TODO{\dots}
%A plan of breaking up the project (who will do what). Note that all team members must make a meaningful/significant contribution (i.e. not just do the reporting/posters).

